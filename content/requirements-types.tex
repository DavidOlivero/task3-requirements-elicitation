\textbf{Paso 2: Tipos de requerimientos}
\begin{enumerate}
    \item \textbf{Requerimientos de información:} Los requerimientos de información son especificaciones detalladas de la información necesaria para satisfacer ciertas necesidades en procesos de toma de decisiones, gestión de proyectos, o cumplimiento de normativas en los sistemas funcionales. Al respecto \cite{yinghuietal2014} mencionan que los requerimientos clave de información en sistemas operativos se estructuran en tres elementos principales: Nodo, que representa una entidad o punto específico dentro del sistema; Actividad, que se refiere a las acciones o procesos que ocurren dentro del marco operativo; y Fase, que identifica diferentes etapas o segmentos del proceso operativo. Teniendo en cuenta que, estos elementos se fundamentan en principios de redes complejas, lo que facilita la comprensión de sus interrelaciones y funciones dentro del sistema.
    \item \textbf{Requerimientos funcionales:} Según lo expresado por \cite{fultonyvandermolen2017} los requerimientos funcionales son especificaciones esenciales que describen cómo debe comportarse un sistema, detallando cómo cada entrada debe generar una respuesta o salida específica o cómo reacciona a un evento temporal. A este respecto se añade que, son considerados uno de los tipos de requerimientos más recomendables, ya que son medibles y verificables, es decir, permiten comprobar su cumplimiento a través de indicadores cuantificables. Además, estos requerimientos están formulados de manera que no dependen del diseño ni de la implementación física del hardware, lo cual facilita su aplicabilidad en diversas etapas del desarrollo.
    \item \textbf{Requerimientos no funcionales:} Mientras que los requerimeintos funcionales expresaban lo que debía hacer el sistema, los requeriminetos no funcionales establecen limitaciones y condiciones específicas para el sistema, restringiendo algunos de sus servicios o funciones. Así pues, estos requerimientos incluyen, por ejemplo, restricciones de tiempo, el tipo de proceso de desarrollo que se debe emplear o la capacidad de almacenamiento necesaria \parencite{prietoychacon2017}.
    \item \textbf{Requerimientos del producto:} Los requerimientos del productos son una combinación entre los requerimientos funioncales y los no funcionales del sistema. Así lo demuestran lo dicho por \cite{maiden2001where} que señalan que, los requerimientos del producto abarcan las expectativas de los usuarios y las restricciones técnicas y de diseño. Por tanto, en el proceso de ingeniería de requerimientos, diferentes interesados colaboran para identificar y definir estos aspectos esenciales, combinando creatividad y análisis sistemático.
    \item \textbf{Requerimientos de usuarios:} Es este caso, se puede definir los requerimientos de usarios como las expectativas que una organización establece para que el sistema pueda satisfacer sus necesidades, ya sea para cumplir objetivos de negocio o para adherirse a normas y regulaciones. Así pues, estos requerimientos responden al \ "qué y para qué" \ del sistema, proporcionando la razón de su existencia y la base sobre la cual debe ser desarrollado \parencite{qbdgroup}.
    \item \textbf{Requerimientos del dominio:} Según lo mencionado por \cite{bjørner2008} los requerimientos del dominio son especificaciones técnicas y conceptuales que definen el contexto en el que se usará un sistema. A este respecto se añade que, estos requerimientos detallan el \ “dominio del problema”\ o el entorno de la aplicación, enfocándose en las características propias de ese contexto, como sus normas, procesos y limitaciones específicas.
    \item \textbf{Requerimientos de calidad:} Al respecto, \cite{eckhardt2016An} referencian que, los requerimientos de calidad son especificaciones no funcionales que establecen las características que un sistema debe tener para cumplir con estándares de rendimiento, seguridad, disponibilidad y usabilidad. Por tanto, a diferencia de los requerimientos funcionales, que describen las tareas que el sistema debe realizar, los requerimientos de calidad se enfocan en cómo debe ser el sistema para asegurar una experiencia confiable y satisfactoria para el usuario, además de cumplir con los criterios de calidad establecidos.
    \item \textbf{Requerimientos de rendimiento:} Los rquerimientos de rendimiento son pautas que determinan el comportamiento esperado de un sistema en aspectos como velocidad, capacidad de respuesta y eficiencia en diferentes niveles de carga. Por ende, estos criterios establecen métricas concretas, como el tiempo máximo de respuesta aceptable y el número de operaciones que el sistema debe manejar en un tiempo específico. Por tanto, son fundamentales para garantizar una experiencia de usuario óptima y un funcionamiento confiable del sistema, incluso en situaciones de alta demanda \parencite{eckhardt2016challenging}.
    \item \textbf{Requerimientos de seguridad:} Según lo mencionado por \cite{espejel2019}, los requerimeintos de seguridad son una categoría específica de los requerimientos funcionales que impactan o influyen directamente en la protección del sistema. En síntesis, este tipo de requerimintos funcionales describe de que manera el sistema debe ser capaz de resguardad los datos y ofrecer una experiencia segura al usuario.
    \item \textbf{Otros tipos de requerimientos:} Además de los  tipos de requerimientos vistos anteriormente, existen otros tales como los requerimientos de usabilidad los cuales especifican cómo de fácil y eficiente debe ser la interacción de los usuarios con el sistema. Adicionalmente, existen tambien los requerimientos de compatibilidad que, aseguran que el producto funcione correctamente en diferentes sistemas operativos, navegadores o plataformas. Además de esto, otro de los requerimientos que influyen en el proceso de desarrollo de software, son los rquerimientos de escalabilidad que, definen la capacidad del sistema para manejar el crecimiento, ya sea en términos de usuarios, transacciones o datos
\end{enumerate}