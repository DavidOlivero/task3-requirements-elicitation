\textbf{Paso 3: Obtención de requerimientos}
\textbf{Caso de estudio:} Caso de estudio: El Gimnasio S\&F (Salud y Fuerza) es una empresa local que ofrece el servicio de entrenamiento y acondicionamiento físico. S\&F cuenta con varias sedes en diferentes lugares de la ciudad y ofrece a sus clientes máquinas de entrenamiento, acompañamiento con instructores cualificados, programas de acondicionamiento, clases de baile entre otros servicios.

El Gimnasio F\&S requiere un sistema de información para gestionar toda la información de sus actividades, incluyendo información de sus clientes, instructores y equipos deportivos.

Para poder identificar los requerimientos del sistema, se hace necesario reconocer los objetivos generales del este. Por tanto, para el caso del Gimnasio S\&F, se pueden identificar los siguientes objetivos:

\begin{itemize}
    \item \textbf{Gestión de Clientes:} Registro y seguimiento de la información personal, progreso de entrenamiento y membresía.
    \item \textbf{Gestión de Instructores:} Control de horarios, perfiles y disponibilidad de instructores.
    \item \textbf{Gestión de Equipos y Máquinas:} Control de mantenimiento, disponibilidad y ubicación de equipos.
    \item \textbf{Programas y Actividades:} Seguimiento de clases de baile, entrenamientos y programas de acondicionamiento.
\end{itemize}

En este orden de ideas, teniendo presentes los objetivos generales del sistema, se debe aplicar alguna de las técnicas de elicitación de requerimientos disponibles. Así pues, para este caso en particular, se utilizará la técnia de revisión de documentos. Para este propósito entonces, se estudiaron dos casos de éxito de otros sistemas de gerencia de gimnasios.

\textbf{Caso We/On con el software virtuagym (\href{https://business.virtuagym.com/es/cliente/we-on/}{Leer caso de éxito aquí}).}

Teniendo en cunta el caso de éxito del gimnasio We/On con la utilización del sofware virtuagym, se pueden identificar los siguientes requerimientos para el sistema del gimnasio S\&F:

\begin{enumerate}
    \item \textbf{Objetivos inspirados por el caso virtuagym:}
    \begin{itemize}
        \item \textbf{Mantener el compromiso y la retención de los clientes:} El caso de virtuagym demostró que la retención de clientes y el mantenimiento del interés pueden lograrse a través de contenido en línea, funciones de comunidad y seguimiento digital.
        \item \textbf{Facilitar el acceso remoto a los servicios:} La creación de una aplicación o portal web para el acceso a planes, horarios y clases en línea es una característica útil, especialmente para S\&F, que cuenta con múltiples sedes.
        \item \textbf{Fortalecer la comunidad del gimnasio:} La integración de funciones de interacción social o una comunidad donde los clientes compartan progresos o participen en desafíos puede ser beneficiosa.
    \end{itemize}
    \item \textbf{Requerimientos clave extraídos del caso:}
    \begin{itemize}
        \item \textbf{Aplicación móvil personalizada:} Al igual que virtuagym, una aplicación personalizada que permita a los clientes consultar su progreso, acceder a planes de entrenamiento y mantenerse conectados con el gimnasio de forma remota puede ser una característica valiosa.
        \item \textbf{Funciones de comunicación y contenido a distancia:} La integración de una plataforma para enviar notificaciones, compartir rutinas y dar soporte remoto puede fortalecer la comunicación entre instructores y clientes.
        \item \textbf{Seguimiento de progreso y salud:} La implementación de la capacidad de los clientes para llevar un registro de su avance físico y de salud, como rutina, peso o IMC, dentro del sistema puede ser una herramienta útil.
    \end{itemize}
    \item \textbf{Condiciones y restricciones del sistema, basadas en el contexto de \\ We/On:}
    \begin{itemize}
        \item \textbf{Entorno seguro y digital:} Es fundamental asegurar que la plataforma incluya medidas de seguridad digital para proteger la información personal y de salud de los clientes.
        \item \textbf{Accesibilidad y usabilidad:} La aplicación o plataforma debe ser intuitiva y adaptable para que tanto los clientes como el equipo de S\&F puedan usarla sin dificultad.
        \item \textbf{Adaptación a situaciones especiales:} El sistema debe ser capaz de funcionar de manera flexible para adaptarse a diferentes escenarios, como la posibilidad de ofrecer clases virtuales o a distancia si surgen nuevas restricciones.
    \end{itemize}
\end{enumerate}

\textbf{Caso Sportium con el software Trainingym (\href{https://trainingym.com/es/stories-sportium}{Leer caso de éxito aquí}).}

Teniendo en cuenta este caso de éxito, se puede identificar que el sistema del Gimnasio S\&F debe cumplir con los siguientes requisitos para garantizar una experiencia óptima para los clientes y el personal administrativo.

\begin{enumerate}
    \item \textbf{Funcionalidad de Gestión de Reservas}
    \begin{itemize}
        \item \textbf{Requerimiento:} Un sistema de reservas que permita a los clientes de S\&F apartar cupo para clases y sesiones de entrenamiento, asegurando su lugar y evitando sobrecarga en los espacios.
        \item \textbf{Propósito:} Mejora la experiencia del cliente al garantizar disponibilidad y permitir planificación, además de optimizar el uso de las instalaciones del gimnasio.
        \item \textbf{Ejemplo de Aplicación:} Al igual que en Sportium, esta funcionalidad puede brindar una experiencia sin fricciones, pues los usuarios no tendrán que preocuparse por la disponibilidad al llegar al gimnasio.
    \end{itemize}
    \item \textbf{Análisis de Datos para Demandas y Preferencias de Clases}
    \begin{itemize}
        \item \textbf{Requerimiento:} Implementación de un módulo de análisis de datos para monitorear la demanda de clases, horarios y preferencias de los clientes, utilizando métricas como la asistencia, las reservas y las calificaciones de los usuarios.
        \item \textbf{Propósito:} Permitir al gimnasio S\&F ajustar su oferta de actividades y horarios de acuerdo con los datos obtenidos, optimizando el uso de los recursos y mejorando la satisfacción del cliente.
        \item \textbf{Ejemplo de Aplicación:} Similar a Sportium, se podría ofrecer una funcionalidad que permita a los usuarios calificar las clases, proporcionando información que ayude al gimnasio a realizar ajustes continuos.
    \end{itemize}
    \item \textbf{Comunicación Instantánea a través de Notificaciones Push}
    \begin{itemize}
        \item \textbf{Requerimiento:} Un sistema de notificaciones push para comunicar cambios de última hora, como cancelaciones de clases, problemas técnicos o modificaciones de horarios.
        \item \textbf{Propósito:} Garantizar que los clientes estén siempre informados de cualquier incidencia, lo que puede reducir las quejas y aumentar la satisfacción general.
        \item \textbf{Ejemplo de Aplicación:} Como lo hace Sportium, S\&F podría enviar notificaciones directamente a los dispositivos de los usuarios para mejorar la eficiencia de la comunicación.
    \end{itemize}
    \item \textbf{Herramientas de Marketing y Medición de Impacto}
    \begin{itemize}
        \item \textbf{Requerimiento:} Módulo de marketing integrado que permita crear, ejecutar y analizar campañas de marketing para promocionar servicios o paquetes, y medir el impacto en tiempo real.
        \item \textbf{Propósito:} Ayudar al gimnasio a evaluar el éxito de sus campañas promocionales y optimizar el uso de sus recursos de marketing.
        \item \textbf{Ejemplo de Aplicación:} Como Sportium, el gimnasio S\&F podría tomar decisiones rápidas sobre las campañas en función de datos en tiempo real, mejorando la eficiencia y la claridad en el retorno de inversión.
    \end{itemize}
    \item \textbf{Soporte Técnico Rápido y Fácil de Usar}
    \begin{itemize}
        \item \textbf{Requerimiento:} Proveer un sistema que facilite la experiencia de usuario tanto para el equipo administrativo como para los clientes, y que tenga un soporte técnico eficiente.
        \item \textbf{Propósito:} Reducir el tiempo de adaptación y resolver problemas rápidamente, contribuyendo a una experiencia de usuario sin complicaciones.
        \item \textbf{Ejemplo de Aplicación:} Como en Sportium, S\&F podría beneficiarse de una interfaz intuitiva y de un soporte técnico confiable para asegurar que el personal y los clientes usen el sistema de manera efectiva.
    \end{itemize}
\end{enumerate}

En este preciso, conciderando los requerimientos conseguidos al estudiar todos los casos de éxito de otros sistemas de gestión de gimnasios obtenidos del análisis de documentos, se pueden clasificar de la siguiente manera:

\begin{enumerate}
    \item \textbf{Requerimientos de Información}
    \begin{itemize}
        \item \textbf{Análisis de datos de demanda y preferencias:} Obtener y analizar datos sobre asistencia, reservas y calificaciones de clases para ajustar la oferta de servicios de acuerdo con la demanda.
        \item \textbf{Seguimiento de progreso y salud:} Registrar y almacenar datos de progreso físico y de salud (como peso, IMC, rutinas) para monitorear la evolución de los clientes.
    \end{itemize}
    \item \textbf{Requerimientos Funcionales}
    \begin{itemize}
        \item \textbf{Gestión de reservas:} Permitir a los usuarios reservar clases y sesiones, garantizando su lugar y optimizando el uso de las instalaciones.
        \item \textbf{Aplicación móvil personalizada:} Proporcionar una app que permita consultar progreso, acceder a planes de entrenamiento y comunicarse con el gimnasio.
        \item \textbf{Comunicación a través de notificaciones:} Enviar notificaciones push para comunicar cambios de última hora o información relevante, como modificaciones en horarios.
        \item \textbf{Marketing y medición de impacto:} Crear y medir campañas de marketing directamente desde el sistema, con indicadores en tiempo real.
    \end{itemize}
    \item \textbf{Requerimientos No Funcionales}
    \begin{itemize}
        \item \textbf{Accesibilidad y usabilidad:} Diseñar la plataforma de manera intuitiva para que sea accesible y fácil de usar para clientes y personal administrativo.
        \item \textbf{Adaptabilidad a situaciones especiales:} Permitir que el sistema funcione de manera flexible y que pueda adaptarse a restricciones o cambios en las modalidades de servicio (como clases en línea).
    \end{itemize}
    \item \textbf{Requerimientos del Producto}
    \begin{itemize}
        \item \textbf{Aplicación móvil personalizada:} Ofrecer una app dedicada que permita al gimnasio tener su propia marca y diseño, promoviendo una experiencia de usuario coherente y personalizada.
        \item \textbf{Fomento de la comunidad:} Incluir una función para crear una comunidad digital que incentive la interacción social y el sentido de pertenencia entre los usuarios.
    \end{itemize}
    \item \textbf{Requerimientos de Usuarios}
    \begin{itemize}
        \item \textbf{Reservas anticipadas:} Brindar a los usuarios la posibilidad de reservar cupo para clases y actividades, asegurando su lugar y evitando sobrecargas en los espacios.
        \item \textbf{Interfaz intuitiva y accesible:} Diseñar una plataforma que sea fácil de usar y que no requiera conocimientos técnicos para la navegación por parte de los clientes y el equipo administrativo.
    \end{itemize}
    \item \textbf{Requerimientos del Dominio}
    \begin{itemize}
        \item \textbf{Seguimiento de salud y rendimiento físico:} Adaptar el sistema para ofrecer funciones específicas de la industria fitness, como monitoreo de métricas físicas y registro de progreso de salud.
        \item \textbf{Gestión y análisis de clases y horarios:} Ajustar la oferta de clases y horarios del gimnasio según el comportamiento y preferencias de los usuarios, típicos en la industria del fitness.
    \end{itemize}
    \item \textbf{Requerimientos de Calidad}
    \begin{itemize}
        \item \textbf{Soporte técnico eficiente:} Proveer un servicio de soporte técnico que asegure una experiencia sin interrupciones y que facilite la resolución de problemas rápidamente.
        \item \textbf{Actualización continua:} Mantener el sistema actualizado con mejoras que puedan ajustarse a las tendencias y necesidades cambiantes de la industria fitness.
    \end{itemize}
    \item \textbf{Requerimientos de Rendimiento}
    \begin{itemize}
        \item \textbf{Notificaciones en tiempo real:} Asegurar que el sistema de notificaciones push funcione de forma inmediata para informar a los usuarios sin demoras.
        \item \textbf{Gestión eficiente de reservas:} Garantizar que el sistema de reservas funcione sin errores o demoras durante periodos de alta demanda.
    \end{itemize}
    \item \textbf{Requerimientos de Seguridad}
    \begin{itemize}
        \item \textbf{Seguridad y privacidad de datos:} Implementar medidas de seguridad para proteger la información personal y de salud de los usuarios, asegurando el cumplimiento de normativas de protección de datos.
        \item \textbf{Entorno digital seguro:} Diseñar un entorno confiable donde los datos sensibles se manejen con protocolos de seguridad, protegiendo la información de accesos no autorizados.
    \end{itemize}
\end{enumerate}